%% Example of use of pythontex in combination with pyCost
\documentclass[11pt]{article}% 
\usepackage{pythontex}
\usepackage{longtable}
\usepackage{lipsum}

\begin{document} 
 
 
\begin{pycode} 
 
import os
from pycost.structure import obra
from pycost.utils import pylatex_utils

obra= obra.Obra(cod="test", tit="Test title")

# Read data from file.
pth= os.path.dirname(__file__)
# print("pth= ", pth)
if(not pth):
    pth= '.'
obra.readFromYaml(pth+'/../../../verif/tests/data/yaml/test_file_05.yaml')

def getPriceTable(codes):
    tabData= obra.getPriceDescriptions(codes)
    retval= pylatex_utils.getTabularDataFromList(tabData)
    return retval.strip()

def getTable(codes):
    retval= '\\begin{longtable}{|c|c|p{9cm}|} \n'
    retval+= '\\hline \n'
    retval+= '\\textbf{Código} & \\textbf{Unidad} & \\textbf{Descripción} \\\\ \n'
    retval+= '\\hline \n'
    retval+= '\\endfirsthead \n'
    retval+= '\\hline \n'
    retval+= '\\textbf{Código} & \\textbf{Unidad} & \\textbf{Descripción} \\\\ \n'
    retval+= '\\hline \n'
    retval+= '\\endhead \n'
    retval+= '\\hline \n'
    retval+= '\\endfoot \n'
    retval+= '\\hline \n'
    retval+= '\\endlastfoot \n'
    retval+= getPriceTable(codes)
    retval+= "\\end{longtable}\n"
    return retval
    
        
\end{pycode}
 
The price table is:

\begin{center}
\begin{small}
    \pyc{tabular_str= getTable(['CANALACO26','CANALACO36','CELAUS','CERSIL','CIMENTA01'])}%
    \pyc{print(tabular_str, file=sys.stdout, flush=True)}%
\end{small}
\end{center}

\lipsum

\end{document}
